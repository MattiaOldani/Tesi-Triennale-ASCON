\section{Conclusioni}

%=======================================================================

\begin{frame}{Risultati ottenuti}

\begin{enumerate}[<+->]
    \item \textbf{Ottimi tempi di esecuzione} --- Gli algoritmi sono molto veloci considerando l'ambiente limitato nel quale sono eseguiti
    \item \textbf{Poco spazio utilizzato} --- Alcune implementazioni riducono drasticamente la grandezza dell'eseguibile, rendendoli ottimi in ambiti dove lo spazio a disposizione è limitato
    \item \textbf{Duttilità} --- Ci sono implementazioni che ottimizzano -- o avvicinano all'ottimo -- entrambi gli aspetti analizzati
\end{enumerate}

\end{frame}

%=======================================================================

\begin{frame}{Sviluppi futuri}

I principali sviluppi futuri si muoveranno in quattro direzioni:
\begin{enumerate}[<+->]
    \item \textbf{Raccolta dati} --- Analisi dei cicli della CPU
    \item \textbf{Board} --- Architetture IoT non testate (ARMv6, ARM neon, ESP32, AVR, eccetera)
    \item \textbf{Plaintext} --- File di grandezza maggiore di 1024 byte (immagini o video)
    \item \textbf{Testing automatico} --- Realizzazione di script che automatizzino il testing usando, ad esempio, l'Arduino IDE dal terminale e non tramite la GUI
\end{enumerate}

\end{frame}
