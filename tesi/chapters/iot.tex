\chapter{Architettura IoT (COMPLETATO)}

\section{Definizione}

L'IoT, o \textit{Internet delle cose}, descrive una rete di dispositivi fisici integrati tra loro tramite sensori, software e la rete, consentendo loro di raccogliere e condividere dati. I dispositivi IoT sono conosciuti anche come \textit{``smart objects''}, e possono variare dai semplici dispositivi \textit{smart home}, come i termostati intelligenti o gli \textit{smartwatch}, fino a macchinari industriali complessi e ai sistemi di trasporto. Una delle ultime novità in ambito IoT rappresenta l'idea delle ``smart city'', basate interamente sulle tecnologie IoT\cite{iot-intro}. \\

\noindent I dispositivi IoT sono molto particolari perché hanno dimensioni molto ridotte, quindi hanno scarsa memoria e processori con potenza computazionale limitata. Queste limitazioni lato hardware devono essere compensate con delle ottimizzazioni lato software tramite algoritmi lightweight, che vengono introdotti nel capitolo successivo. \\

\noindent Nonostante ciò, questa architettura é in rapida crescita ed é sempre più presente nella vita quotidiana delle persone.

\section{Contesto di utilizzo}

\noindent Lo schema IoT consente ai dispositivi di comunicare tra loro e con altri dispositivi abilitati all'uso di internet. Questo permette di creare una rete di dispositivi interconnessi che possono scambiare dati e svolgere varie attività autonomamente, come ad esempio: \begin{itemize}
    \item monitorare le condizioni ambientali;
    \item gestire i modelli di traffico con automobili intelligenti, come la guida automatica presente nei prodotti Tesla;
    \item controllare macchine e processi nelle fabbriche;
    \item tracciare delle spedizioni nei magazzini.
\end{itemize}
I settori che possono trarre più vantaggio dall'IoT sono i seguenti\cite{iot-use}: \begin{enumerate}
    \item manifatturiero: si monitora la linea di produzione per verificare quando viene compromessa e si permette una rapida gestione degli asset;
    \item automobilistico: situazione simile alla precedente, ovvero la verifica della compromissione degli asset, che viene poi segnalata all'utilizzatore durante l'uso del veicolo; 
    \item trasporto e logistica: rilevazione degli inventari per il monitoraggio delle spedizioni o della temperatura, soprattutto su prodotti alimentari e farmaceutici;
    \item sanità: rilevazione dell'esatta posizione delle risorse di assistenza, oltre al monitoraggio real time delle condizioni dei pazienti.
\end{enumerate}
\noindent Le due liste appena presentate sono molto riduttive: il mondo IoT é vario e capace di svolgere quasi tutte le operazioni possibili, e i settori che ne giovano sono molti di più rispetto a quelli presentati.

\section{IoT vs sistemi embedded}

Un primo confronto possibile é tra l'IoT e i sistemi embedded. Spesso i due termini vengono confusi e utilizzati come se fossero la stessa cosa, ma questo non é vero: infatti, i sistemi embedded, se integrati all’interno di oggetti o sistemi informatici più complessi, costituiscono una possibile soluzione IoT, ma non necessariamente lo sono. Quest'ultimo é il caso dei sistemi progettati per ottenere uno scambio “chiuso”, cioè di dati e informazioni tra dispositivi senza interazione con l’ambiente. Non basta quindi, per un sistema embedded, essere dotato di micro-controllori con sensori e altri apparecchi fisici per essere considerato IoT\cite{iot-embedded}.

\section{IoT VS macchine ad alte prestazioni}

Un secondo confronto é invece quello tra l'IoT e le macchine ad alte prestazioni, dette anche HCP (\textit{High-Performance Computing})\cite{hcp}, riassunto nella tabella successiva.

\begin{table}[H]
    \centering
	\begin{tabular}{|m{0.22\textwidth}<{\centering}||m{0.39\textwidth}<{\centering}|m{0.39\textwidth}<{\centering}|}
		\hline
		Aspetto & IoT & HCP \\
        \hline \hline
        Definizione. & Rete di dispositivi fisici connessi tramite la rete internet, la quale permette lo scambio dei dati raccolti tramite sensori o simili. & Tecnologia che utilizza cluster di processori per elaborare enormi quantità di dati, detti anche \textit{big data}. \\ 
        \hline
        Dimensioni. & Molto ridotte: l'ordine di grandezza é quello dei sensori, ovvero i principali dispositivi IoT. & Molto grandi: sono utilizzati cluster di processori anche da 100.000 nodi. \\
        \hline
        Potenza computazionale. & Limitata viste le ridotte dimensioni. & Enorme vista la grande quantità di processori che possono lavorare in parallelo. La velocità é quasi un milione di volte superiore rispetto ai classici sistemi desktop. \\
        \hline
        Flusso di esecuzione. & Seriale. & Parallelo di massa. \\
        \hline
        Ambito di utilizzo. & Raccogliere e inviare dati tramite la rete per poterli analizzare. & Machine Learning, Artificial Intelligence, rendering grafico, sanità, genomica, finanza e trading, governo e difesa. \\
		\hline
    \end{tabular}
    \caption{Differenze tra IoT e HCP.}
\end{table}

\section{Perché scegliere il mondo IoT}

Il mondo IoT porta una serie di benefici alla vita delle persone: le soluzioni sviluppate rendono la vita più facile e conveniente in vari ambiti, come quello sanitario e dell'istruzione. \\

\noindent Alcuni benefici di questa tecnologia sono\cite{iot-vantaggi}: \begin{enumerate}
    \item raccolta efficiente dei dati: utile in settori come sanità e finanze per tracciare le decisioni sugli acquisti e le tendenze di vendita. I dati sono poi usati per migliorare la gestione degli inventari e rilevare i comportamenti del cliente;
    \item controllo e automazione: si permette uno stile di vita controllabile ``con un tocco'', ad esempio tramite dispositivi intelligenti come lampadine, macchine per il caffé o, in generale, dispositivi di uso quotidiano;
    \item accesso in tempo reale alle informazioni: si offre accesso immediato alle informazioni, risultando preziosi in settori come la sanità, le imprese e le applicazioni quotidiane. Soprattutto in ambito medico é utile: ad esempio é possibile monitorare la salute di un paziente in tempo reale, risultando cruciale nel fornire assistenza medica tempestiva;
    \item miglioramento dell'efficienza: i sistemi IoT operano autonomamente, quindi con meno interferenza umana si ha un aumento dell'efficienza e una minore dipendenza dal lavoro umano;
    \item tracciamento degli asset: permette il tracciamento dei prodotti all'interno di un'impresa o di un sistema di gestione della logistica. Il tracciamento manuale degli asset è laborioso e richiede tempo, ma può essere semplificato attraverso l'applicazione di tecnologie IoT come codici a barre e tag RFID;
    \item aumento della produttività: in relazione al secondo punto, grazie ai dispositivi smart e intelligenti gli utenti possono semplificare varie attività domestiche usando comandi vocali o applicazioni;
    \item sicurezza: si possono monitorare a distanza i propri asset di valore, come veicoli o oggetti di collezione, oppure tracciare la posizione dei figli comodamente dal telefono;
    \item miglioramento del coinvolgimento del cliente: sfruttando i dati dei clienti é possibile la personalizzazione delle esperienze, migliorando la convenienza e consentendo interazioni in tempo reale;
    \item utilizzo efficiente delle risorse: il tracciamento dello stato delle risorse, come attrezzature e macchinari, consente l'identificazione di inefficienze. La manutenzione predittiva nel settore industriale può prevedere guasti delle macchine, riducendo i tempi di inattività e ottimizzando l'allocazione delle risorse per la manutenzione.
\end{enumerate}

\section{Sfide del mondo IoT}

Il mondo IoT deve affrontare una serie di sfide. Alcune di queste sono\cite{iot-challenge-1}\cite{iot-challenge-2}: \begin{enumerate}
    \item sicurezza e la privacy: con l'aumentare della diffusione dei dispositivi IoT la sicurezza e la privacy diventano sempre più importanti. Molti dispositivi IoT sono vulnerabili ad attacchi su più livelli dello stack di rete, quindi bisogna ricorrere a tecniche di anonimato e crittografia per proteggere le enormi quantità di dati, anche personali, che sono raccolte;
    \item interoperabilità: i dispositivi IoT di diversi produttori spesso utilizzano standard e protocolli diversi, rendendo difficile la comunicazione. Una possibile soluzione é l'utilizzo di interfacce standard, ma la creazione di specifici standard e la successiva imposizione é un compito arduo e praticamente impraticabile;
    \item sovraccarico dei dati: i dispositivi IoT generano enormi quantità di dati, che possono sovraccaricare le aziende non preparate a gestirli;
    \item costi e complessità: Implementare un sistema IoT può essere costoso e complesso, richiedendo investimenti significativi in hardware, software e infrastruttura. La manutenzione invece richiede competenze e esperienza specializzate;
    \item sfide normative e legali: Con l'aumentare della diffusione dei dispositivi IoT, stanno emergendo sfide normative e legali. Le aziende devono conformarsi a varie normative sulla protezione dei dati, sulla privacy e sulla sicurezza informatica, che possono variare da paese a paese;
    \item scalabilità: all'aumentare del numero di dispositivi bisogna garantire una connettività fluida, una gestione efficace dei dati e prestazioni complessive ottime, ma questa é complicato se non si utilizzano tecnologie come componenti modulari, bilanciatori di carico e sistemi distribuiti.
\end{enumerate} 
