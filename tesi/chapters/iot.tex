\chapter{Architettura IoT}

\section{Definizione}

L'\textbf{IoT}, dall'inglese \textit{Internet of Things} (ossia \textit{Internet delle cose}), descrive una rete di dispositivi fisici integrati tra loro tramite sensori, software e rete, consentendogli di raccogliere e condividere dati. I dispositivi IoT sono conosciuti anche come \textit{smart objects}, e possono variare dai semplici dispositivi \textit{smart home}, come i termostati intelligenti o gli \textit{smartwatch}, fino a macchinari industriali e ai sistemi di trasporto. Una delle ultime novità in ambito IoT rappresenta l'idea delle \textit{smart cities}, basate interamente sulle tecnologie IoT\cite{iot-intro}. \\

\noindent I dispositivi IoT sono degni di interesse perché presentano dimensioni molto ridotte, pertanto hanno scarsa memoria e processori con potenza computazionale limitata. Queste limitazioni lato hardware devono essere compensate con delle ottimizzazioni lato software mediante algoritmi lightweight, che vengono introdotti nel capitolo successivo (vedi \rif[Capitolo]{ch:cryptography}). \\

\noindent A fronte di tutte queste problematiche, questa architettura è in rapida crescita ed è sempre più presente nella vita quotidiana delle persone.

\section{Contesto di utilizzo}

\noindent Lo schema IoT consente ai dispositivi abilitati all'uso di internet di comunicare tra loro, scambiare dati e svolgere varie attività in maniera autonoma – ad esempio:
\begin{itemize}
    \item Monitorare le condizioni ambientali;
    \item Gestire i modelli di traffico con automobili intelligenti, come la guida automatica presente nei prodotti Tesla \copyright;
    \item Controllare macchine e processi nelle fabbriche;
    \item Tenere traccia delle spedizioni nell'\textit{e-commerce}.
\end{itemize}
I settori che possono trarre più vantaggio dall'IoT sono i seguenti\cite{iot-use}:
\begin{itemize}
    \item \textbf{Manifatturiero} --- Si monitora la linea di produzione per verificare quando viene compromessa e si permette una rapida gestione degli asset;
    \item \textbf{Automobilistico} --- Situazione simile alla precedente, ovvero la verifica della compromissione degli asset, che viene poi segnalata all'utilizzatore durante l'uso del veicolo; 
    \item \textbf{Trasporto e logistica} --- Rilevazione degli inventari per il monitoraggio delle spedizioni o della temperatura, soprattutto nei confronti di prodotti deperibili come gli alimentari e i farmaceutici;
    \item \textbf{Sanità} --- Rilevazione dell'esatta posizione delle risorse di assistenza, oltre al monitoraggio real time delle condizioni dei pazienti.
\end{itemize}
\noindent Le due liste appena presentate sono molto riduttive: il mondo IoT è vario e capace di svolgere quasi tutte le operazioni possibili e i settori che ne giovano sono in realtà molti di più rispetto a quelli presentati.

\section{IoT Vs. sistemi embedded}

Un primo confronto possibile è tra l'IoT e i \textbf{sistemi embedded}. Spesso i due termini vengono confusi e utilizzati come se fossero la stessa cosa, ma questo non è vero: i sistemi embedded, se integrati all’interno di oggetti o sistemi informatici più complessi, costituiscono una possibile soluzione IoT, ma \textit{non necessariamente} lo sono. \\

\noindent Quest'ultimo è il caso dei sistemi progettati per ottenere uno scambio ``chiuso'', cioè di dati e informazioni tra dispositivi senza interazione con l’ambiente. Non è sufficiente, dunque, per un sistema embedded, essere dotato di microcontrollori con sensori e altri apparecchi fisici per essere considerato IoT\cite{iot-embedded}.

\section{IoT Vs. macchine ad alte prestazioni}

Un secondo confronto è invece quello tra l'IoT e le \textbf{macchine ad alte prestazioni}, dette anche HCP (\textit{High-Performance Computing})\cite{hcp}, riassunto nella \rif[Tabella]{tab: tabella_iot_hcp} (vedi sotto).

\begin{table}[H]
    \centering
	\begin{tabular}{|m{0.22\textwidth}<{\centering}||m{0.39\textwidth}<{\centering}|m{0.39\textwidth}<{\centering}|}
		\hline
		\textbf{Aspetto} & \textbf{IoT} & \textbf{HCP} \\
        \hline \hline
        Definizione & Rete di dispositivi fisici connessi tramite la rete internet, la quale permette lo scambio dei dati raccolti tramite sensori o simili & Tecnologia che utilizza cluster di processori per elaborare enormi quantità di dati, detti anche \textit{big data} \\ 
        \hline
        Dimensioni & Molto ridotte: l'ordine di grandezza è quello dei sensori, ovvero i principali dispositivi IoT & Molto grandi: sono utilizzati cluster di processori anche da 100000 nodi \\
        \hline
        Potenza computazionale & Limitata viste le ridotte dimensioni & Enorme vista la grande quantità di processori che possono lavorare in parallelo La velocità è quasi un milione di volte superiore rispetto ai classici sistemi desktop \\
        \hline
        Flusso di esecuzione & Seriale & Parallelo di massa \\
        \hline
        Ambito di utilizzo & Raccogliere e inviare dati tramite la rete per poterli analizzare & Machine learning, artificial intelligence, rendering grafico, sanità, genomica, finanza e trading, governo e difesa \\
		\hline
    \end{tabular}
    \caption{Differenze tra IoT e HCP.}
    \label{tab: tabella_iot_hcp}
\end{table}

\section{Perché scegliere il mondo IoT}

Il mondo IoT è in grado di migliorare la vita di tutti i giorni, semplificando varie operazioni in diversi ambiti, come quelli sanitario e dell'istruzione. \\

\noindent Alcuni benefici di questa tecnologia sono\cite{iot-vantaggi}:
\begin{enumerate}
    \item \textbf{Raccolta efficiente dei dati} --- In settori come sanità e finanza è utile per tenere traccia delle decisioni sugli acquisti e le tendenze di vendita. I dati sono poi usati per migliorare la gestione degli inventari e rilevare i comportamenti del cliente;
    \item \textbf{Controllo e automazione} --- Si permette uno stile di vita controllabile ``con un tocco'', ad esempio tramite dispositivi intelligenti come lampadine, macchine per il caffè o, in generale, dispositivi di uso quotidiano;
    \item \textbf{Accesso in tempo reale alle informazioni} --- Si offre accesso immediato alle informazioni, particolarmente utile in settori come la sanità. Soprattutto in ambito medico è utile: ad esempio è possibile monitorare la salute di un paziente in tempo reale, elemento che risulta cruciale nel fornire assistenza tempestiva;
    \item \textbf{Miglioramento dell'efficienza} --- I sistemi IoT operano autonomamente, quindi si elimina la dipendenza dal lavoro umano in parallelo ad un aumento dell'efficienza;
    \item \textbf{Tracciamento degli asset} --- Permette il tracciamento dei prodotti all'interno di un'impresa o di un sistema di gestione della logistica. Il tracciamento manuale degli asset è laborioso e richiede tempo, ma può essere semplificato attraverso l'applicazione di tecnologie IoT come codici a barre e tag RFID;
    \item \textbf{Aumento della produttività} --- In relazione al secondo punto, grazie ai dispositivi smart e intelligenti gli utenti possono semplificare varie attività domestiche usando comandi vocali o applicazioni;
    \item \textbf{Sicurezza} --- Si possono monitorare a distanza i propri asset di valore, come veicoli oppure oggetti di collezione, o anche comodamente tracciare la posizione dei figli dal telefono;
    \item \textbf{Miglioramento del coinvolgimento del cliente} --- Sfruttando i dati dei clienti è possibile la personalizzazione delle esperienze, migliorando la convenienza e consentendo interazioni in tempo reale;
    \item \textbf{Utilizzo efficiente delle risorse} --- Il tracciamento dello stato delle risorse, come attrezzature e macchinari, consente l'identificazione di inefficienze. La manutenzione predittiva nel settore industriale può prevedere guasti delle macchine, riducendo i tempi di inattività e ottimizzando l'allocazione delle risorse per la manutenzione.
\end{enumerate}

\section{Sfide del mondo IoT}

Il mondo IoT deve affrontare una serie di sfide. Alcune di queste sono\cite{iot-challenge-1}\cite{iot-challenge-2}:
\begin{enumerate}
    \item \textbf{Sicurezza e la privacy} --- Con l'aumentare della diffusione dei dispositivi IoT la sicurezza e la privacy diventano sempre più importanti. Molti di essi sono vulnerabili ad attacchi su più livelli dello stack di rete, quindi bisogna ricorrere a tecniche di anonimato e crittografia per proteggere le enormi quantità di dati raccolte;
    \item \textbf{Interoperabilità} --- I dispositivi IoT di diversi produttori spesso utilizzano standard e protocolli diversi, rendendo difficile la comunicazione. Una possibile soluzione è l'utilizzo di interfacce standard, ma la creazione di queste ultime, la loro implementazione e la successiva adozione da parte dell'intero panorama informatico ne rende praticamente impossibile l'attuazione;
    \item \textbf{Sovraccarico dei dati} --- I dispositivi IoT generano enormi quantità di dati, che possono sovraccaricare le aziende impreparate a gestirli;
    \item \textbf{Costi e complessità} --- Implementare un sistema IoT può essere costoso e complesso, richiedendo investimenti significativi in hardware, software e infrastruttura. La manutenzione invece richiede competenze ed esperienza specializzate;
    \item \textbf{Sfide normative e legali} --- Con l'aumentare della diffusione dei dispositivi IoT stanno emergendo sfide normative e legali. Le aziende devono conformarsi a varie normative sulla protezione dei dati, sulla privacy e sulla sicurezza informatica, che possono variare da Paese a Paese;
    \item \textbf{Scalabilità} --- All'aumentare del numero di dispositivi bisogna garantire una connettività fluida, una gestione efficace dei dati e prestazioni complessive ottime, ma questo è complicato se non si utilizzano tecnologie come componenti modulari, bilanciatori di carico e sistemi distribuiti.
\end{enumerate} 
