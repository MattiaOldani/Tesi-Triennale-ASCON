\prefacesection{Ringraziamenti}

Scrivo queste parole in conclusione di un percorso lungo e pieno di difficoltà, ma che grazie a moltissime persone è finito nel migliore dei modi: mi sono laureato. Se mi avessero detto, a giugno 2020, che mi sarei laureato tre anni e mezzo dopo non ci avrei mai creduto. Mi ero appena ritirato da Matematica, avevo passato il TOLC per entrare ad Informatica ma non riuscivo ad immatricolarmi. La prima persona che ringrazio è la signora della segreteria che mi ha tenuto compagnia per 45 minuti per cercare di risolvere i miei problemi, grazie a lei sono qui e non al McDonald di Caravaggio a imbustare nuggets e imprecare verso i maranza. \\

\noindent Ringrazio profondamente il mio relatore, il Prof. Andrea Visconti, che mi ha assegnato un lavoro di tirocinio molto interessante e attuale sul quale lavorare, e che si è sempre reso disponibile per rispondere a dubbi e problemi con incontri -- alcuni molto \textit{esotici} come quello su Zoom in metro -- e una serie infinita di mail. Le sue lezioni di Crittografia e Teoria dell'Informazione mi hanno trasmesso la passione per la crittografia e tutto ciò che c'è di teorico nel mondo informatico. \\

\noindent Non posso non ringraziare la banda di matti che ho conosciuto nel mio percorso universitario. Dopo la brutta esperienza a Matematica, dove ho fatto fatica a integrarmi con i compagni, la paura di fare la stessa fine anche qua era alta, ma così non è stato. Ci siamo conosciuti (\textit{quasi}) tutti su Telegram e Discord, ma questo era quello che offriva la pandemia. E forse, tutto questo è stato un bene: conoscendomi, a fatica avrei fatto amicizia con così tanta gente entrato in un'aula con dentro 250 persone. Ringrazio quindi Aceti, Albi, Alessia, Armani, Asaf, Buso, Caro, Ceri, Edu, Faro, Frido, Gigi, Mangio (\textit{quello vero}), Mangio (\textit{quello falso}), Manto, Mirko (\textit{er}) Faina, Mirko Seghezzi, Moro, Recca, Sarti, Silvio, Tella, Vincenzo, Yeger e Yon. Non potevo chiedere compagni migliori, mi avete accolto dal primo momento e ora siete ancora qua, con me, per festeggiare questo traguardo. \\

\noindent Ringrazio Gaetano e Charif, gli unici amici che ho avuto a Matematica, grazie per tutte le ore passate in biblioteca a studiare per gli esami (\textit{passati 1 su 8}) e per le pizze in Piazza Leo tra una lezione e l'altra. Ringrazio anche tutte le persone che ho conosciuto grazie ai miei compagni di università, soprattutto Samu, con il quale ho condiviso una vacanza pazzissima in Croazia, e Grazia, che ha segnato nel bene e nel male il mio percorso durante tutto il secondo anno. \\

\noindent Ringrazio i miei compagni di classe delle superiori, quelli che sono rimasti in contatto con me anche dopo la maturità, con i quali mi sono incontrato ogni tanto negli anni per rivivere ``i bei vecchi tempi''. Ringrazio quindi Botta, Ceres, Elena, Ferro, Gara, Mangio (\textit{il terzo, unico e inimitabile}) e Vaila. Ringrazio anche Antonio, il Gargiu, che so che da lassù mi sta guardando e sta tifando per me. Sempre delle superiori voglio ringraziare il Prof. Bellavita, per avermi trasmesso la passione per l'Informatica, la Prof.ssa Bolzoni e, successivamente, la Prof.ssa Delmari, per avermi reso dipendente dalla Matematica, e il Prof. Bardelli, perché mi manda ogni anno gli auguri di Natale e Pasqua. \\
    
\noindent Ringrazio tutti i miei amici, compaesani e non, che mi conoscono da una vita e non sono ancora finiti in una clinica psichiatrica per colpa mia. Siete degli amici fantastici, ci siete sempre stati per me e spero di avervi per sempre nella mia vita. In questi anni mi siete stati vicini quando alcuni esami non andavano bene ed esultavate con me quando invece andavano bene, parte di questo traguardo è merito vostro. Ringrazio quindi Andrea e Carol, Adele e Rek, Arianna, Lisa, Deep, Biga, Delia, Chiara, Francesca, Laura, Davide e \st{Elena} Martina, Luca Maestri, Manpreet, Miriam e RSP. \\

\noindent In particolare, ringrazio i Roggiani, i miei fratelli, Luca (\textit{RR19}) e Mastro (\textit{RR17}), grazie per tutto quello che fate per me ogni giorno, per non abbandonarmi in palestra ad orari improponibili, per rendermi la persona più felice del mondo e per farmi sentire a casa ogni volta che sono con voi. Ringrazio il Tunet, che assieme al Ciapa mi liberava la mente ogni sera per 2h durante il periodo delle zone Power Ranger e mi ha tenuto compagnia per tutto il periodo universitario tra treni cancellati o perennemente in ritardo. Ringrazio Vittorio, unico (\textit{polipopi-ingegnere}) informatico che la mattina sul treno mi ascolta e non fa una faccia schifata perché non capisce quello che dico, autore del 99\% delle correzioni fatte nella mia tesi, \textbf{é} stato importantissimo nella parte finale del mio percorso. Ringrazio Giacomo, il mio gymbro, il mio cioccolatino ripieno, il mio pookie, grazie per non avermi abbandonato ogni volta che volevo andare ad allenarmi ad orari proibitivi, per aver ascoltato (\textit{e non aver capito}) ogni argomento che facevo a lezione o quello che facevo durante il tirocinio, per tutti gli sgarri fatti dopo 2h chiusi in palestra con Chiara che aspettava solo di tornare a casa, per tutti i concerti che fai ogni tre secondi dove crei ogni volta una parodia diversa, per essere il giocatore più scarso di Clash Royale, grazie veramente per tutto, \textit{you are my sunshine}. Ringrazio Don Emanuele, per tutte le volte che mi ha aiutato, supportato e ``impegnato'' le giornate tra muratori, contabilità e spostamento mobili. Ringrazio il Gippe, che appena ha saputo della mia iscrizione a Informatica ha subito cercato di indirizzarmi sui binari giusti e tifava per me ad ogni esame. Ringrazio tutti gli animatori del Grest, che durante la sessione estiva mi alleggerivano le giornate pesanti di studio facendomi divertire come solo loro sono capaci. Ringrazio tutti i miei gymbro della palestra di Treviglio, soprattutto Francesco, che non vede l'ora di avere una copia della mia tesi per fare una cosa che solo lui sa.  \\

\noindent Ringrazio Martina, la mia ragazza, grazie per esserci stata in questi ultimi anni, per il continuo supporto che mi dai ogni giorno, per sopportare ogni cosa stupida che faccio, per tutte le esperienze fantastiche che abbiamo fatto assieme, per aver ascoltato ogni mio audio di 25 minuti dove ripetevo allo sfinimento gli argomenti dei vari esami, per essermi stata vicina ogni volta che piangevo perché un esame non andava bene o accumulavo troppa ansia, per aver esultato ogni volta che ``HO SISTEMATO IL CODICE, ORA FUNZIONA'' (\textit{avevo dimenticato una virgola, non so programmare}), grazie per essere semplicemente te. \\

\noindent Infine, ringrazio la mia famiglia. Ringrazio mamma Barbara e papà Marco per tutte le volte che mi hanno accompagnato nelle stazioni della bassa cremasca per prendere il treno, per tutte le volte che mi hanno aiutato preparandomi il pranzo o la cena quando ero in ritardo con i miei impegni, per il continuo e immancabile supporto ogni volta che si avvicinava la sessione, per tutto quello che han fatto per me in questi 23 anni, per avermi dato una seconda opportunità dopo la brutta esperienza a Matematica, grazie veramente, so che molto spesso non lo dimostro, ma sono orgogliosissimo di avere due genitori così fantastici. Ringrazio mia sorella Eva, per tutte le volte che ha messo la musica a volume 100 quando dovevo studiare, per tutti i ``mi accompagni di qua?'' ogni volta che avevo un impegno immediato da sbrigare, per tutti i ``io esco alle 3 dalla disco, vieni a prendermi?'' ogni volta che volevo dormire, ma anche per tutti i bei momenti che abbiamo passato assieme negli ultimi anni. Vi ringrazio di cuore, vivere con me non è facile visto che sono molto chiuso e sto molto sulle mie, ma sappiate che vi voglio veramente bene e sono grato di ogni momento passato assieme. Ringrazio i nonni Antonia, Antonio, Carolina e la bisnonna Carla. Ringrazio gli zii Annalisa, Tiziana, Wilma, Eleonora e Simone, Goffredo e Jessica. E per concludere, ringrazio i fantastici cugini Cristiano, Federico, Giorgia, Gianella, Elisa e Alessia. \\

\noindent Ringrazio tutti di cuore, grazie per esserci stati, e grazie a tutti quelli che ci saranno nei miei prossimi traguardi.
