\prefacesection{Ringraziamenti}

Quelli che state leggendo sono i veri ringraziamenti della mia tesi. La prima versione che ho scritto è quella che ho consegnato di fretta entro l'8 aprile (perché mai mi so organizzare con i tempi), ma non sono quelli che avrei voluto scrivere veramente. \\

\noindent Scrivo queste parole in conclusione di un percorso lungo e pieno di difficoltà, ma che grazie a moltissime persone è finito nel migliore dei modi: mi sono laureato. Se mi avessero detto, a giugno 2020, che mi sarei laureato tre anni e mezzo dopo non ci avrei mai creduto. Mi ero appena ritirato da Matematica, avevo passato il TOLC per entrare ad Informatica ma non riuscivo ad immatricolarmi. La prima persona che ringrazio è la \textbf{signora della segreteria} che mi ha tenuto compagnia per 45 minuti per cercare di risolvere i miei problemi, grazie a lei sono qui e non al McDonald di Caravaggio a imbustare nuggets e imprecare verso i maranza. \\

\noindent Ringrazio profondamente il mio relatore, il \textbf{Prof. Andrea Visconti}, che mi ha assegnato un lavoro di tirocinio molto interessante e attuale sul quale lavorare, e che si è sempre reso disponibile per rispondere a dubbi e problemi con incontri -- alcuni molto \textit{esotici} come quello su Zoom in metro -- e una serie infinita di mail. Le sue lezioni di Crittografia e Teoria dell'Informazione mi hanno trasmesso la passione per la crittografia e tutto ciò che c'è di teorico nel mondo informatico. \\

\noindent Non posso non ringraziare la banda di matti che ho conosciuto nel mio percorso universitario. Dopo la brutta esperienza a Matematica, dove ho fatto fatica a integrarmi con i compagni, la paura di fare la stessa fine anche qua era alta, ma così non è stato. Ci siamo conosciuti (\textit{quasi}) tutti su Telegram e Discord, ma questo era quello che offriva la pandemia. E forse, tutto questo è stato un bene: conoscendomi, a fatica avrei fatto amicizia con così tanta gente entrato in un'aula con dentro 250 persone. \\

\noindent Iniziamo. Ringrazio \textbf{Aceti}, capo supremo del network, affiliato a ogni possibile codice amico esistente. Ringrazio \textbf{Albi}, un'amicizia nata con il volantinaggio e che oggi va avanti a suon di "Oldani smettila di crescere per favore". Ringrazio \textbf{Alessia}, una delle poche ragazze conosciute in questo ambiente fallico-centrico, autrice di pazzi festini in casa e di spritz in piazza Leo. Ringrazio \textbf{Armani}, mio fratello, una persona fantastica e inimitabile, l'ho capito quando insieme ad Alessia ha deciso di comprarmi una mini tortina per il mio compleanno, e ci conoscevamo da poco meno di due mesi; insieme ne abbiamo passate tante, feste, esami passati e non, una vacanza pazza in Croazia, e spero di passarne ancora tante assieme. Ringrazio \textbf{Asaf}, abbiamo frequentato poco l'università assieme ma ci siamo sentiti costantemente per preparare esami e insultare la segreteria. Ringrazio \textbf{Buso}, compagno di giri in moto e grandi chiaccherate su Discord, ha un unico difetto: è bresciano. Ringrazio \textbf{Caro}, anche se all'inizio lo insultavo perché mi gufava il parziale di Continuo, piano piano siamo diventati amiconi, fino a diventare segretamente fidanzati (Martina scusa). Ringrazio \textbf{Ceri}, il mio uomo, mio fratello, mi fermo con gli epiteti perché quelli che ci scriviamo in chat rimangono in chat; amicizia iniziata per puro caso grazie all'esame di Continuo e alla festa di Silvio di qualche giorno dopo, dove mi ha tenuto compagnia in macchina da Treviglio fino a Cuggiono; non esiste un giorno dove non lo ringrazio per tutte le volte che mi ha accolto nella sua compagnia per uscire quando non avevo nessuno, per la devastante vacanza in Croazia, per tutte le feste che abbiamo passato dormendo per terra, per tutte le lezioni passate a giocare a Clash Royale, veramente grazie, ora ti aspetto su Brawl Stars. Ringrazio \textbf{Edu}, il mio secondo fratello, all'inizio probabilmente mi odiava perché quando ci siamo conosciuti per la prima volta in presenza mi salutava a fatica, ma poi tra Silab, aperitivi in piazza Leo, feste a casa di Silvio, gallette di Alex Theory e tanto tanto altro siamo diventati quello che siamo ora; lo ringrazio per gli insulti giornalieri che mi lancia, ma soprattutto per tutte le belle parole che ha speso per me in questi anni, veramente grazie. Ringazio \textbf{Faro}, compagno di Discord fino a tarda notte nel periodo della quarantena, pessimo navigatore in macchina ma sempre un piacere quando lo incontro dopo tanto tempo che non lo vedo. Ringrazio \textbf{Frido}, abbiamo condiviso pochi momenti in università ma sono stati preziosi e importanti, lo ringrazio soprattutto per non aver cacciato di casa me e gli altri quando sotto il suo appartamento ci siamo messi a fare un concerto. Ringrazio \textbf{Gigi}, mio fratello informatico teorico, compagno di insulti a Manto perché è un babbo e compagno di Magistrale che spero di avere al mio fianco fino alla fine. Ringrazio \textbf{Mangio}, \textit{quello vero}, mio fratello ASSOLUTO, praticamente un google vivente; non penso di aver mai conosciuto una persona così intelligente ma pratica allo stesso tempo, vicino di banco in ogni singola lezione che abbiamo fatto in triennale, e ora anche in magistrale (GPU computing goes brrrrr); lo ringrazio per aver finito quel liquore maledetto e poi aver fatto la notte sul water, per tutte le feste che abbiamo fatto insieme e per tutti gli aiuti che mi ha dato, sei una persona fantastica. Ringrazio \textbf{Mangio}, \textit{quello falso}, mio fratello dai monti bergamaschi, insultato e bistrattato dal primo momento perché avevamo già un Mangio in compagnia, probabilmente la laurea l'ha comprata su AliExpress visto il risultato scarsissimo ottenuto alle domande della tua festa di laurea; lo ringrazio perché ogni volta che dicevo "dai bagai, birretta" lui non si tirava mai indietro, per quella volta che si è offerto di ospitarmi a casa sua al mio compleanno, per le partite a basket dove PALESEMENTE gli ho fatto il culo, per tutto quello che abbiamo condiviso, grazie fratello. Ringrazio \textbf{Manto}, fratello pipino, prima persona che ho conosciuto in università, non citerò niente di quello che ci scriviamo se non voglio finire dietro le sbarre in tempo zero, lo ringrazio perché mi ricorda sempre che esiste un simp più grande di me (ciao \textbf{Elena}). Ringrazio \textbf{Mirko (\textit{er}) Faina}, mio fratello Black Russian (sfortunatamente, cit non per pochi), ormai addottato dal Poul, in università abbiamo passato tanto tempo di qualità assieme, soprattutto tutte quelle volte dove dovevo svegliarti perché ti eri addormentato a lezione. Ringrazio \textbf{Mirko Seghezzi}, conosciuto grazie al Ceri, tanto tempo condiviso tra vacanza pazza in Croazia, Silab, caffé pre-lezione e grandi festoni; per non essere lapidato, non dirò come ha accolto me e Mangio alla sua ultima festa di compleanno. Ringrazio \textbf{Moro}, conosciuto tramite il Recca e ora mio compagno di fiducia a Informatica Teorica, grazie alla sua penna magica sono riuscito a finire l'esame di Archi2 in tempo. Ringrazio il \textbf{Recca}, scambiato per un quarantenne la prima volta che ho visto la sua foto profilo, si è poi rivelato una persona fantastica sulla quale contare ogni volta che avevo bisogno di una mano, un tirocinio fatto (circa) insieme, lo ringrazio per tutti i bei momenti passati insieme, anche fe non fo fe fo fegliere un momento migliore. Ringrazio \textbf{Sarti}, fratello palestrato, compagno di grandi pranzi il primo anno tra riso-pollo-broccoli e banane spappolate, menomale su telegram esiste il 2x per gli audio. Ringrazio \textbf{Silvio}, fratello che abita in culo al mondo e che puntualmente faceva 12 feste all'anno, non ti ringrazio per tutta la benzina che ho usato ma ti ringrazio per tutte le serate passate su Discord a convincermi a comprare Overwatch, per i giorni passati in Silab a studiare per gli esami e per cercare ogni giorno di convincermi che il corso di Logica Matematica sia un bel corso. Ringrazio \textbf{Tella} per le grandi mangiate da fratello Luca e per gli infiniti aperitivi in piazza Leo nella infamissima spritzeria. Ringrazio \textbf{Vincenzo}, conosciuto durante l'ultimo semestre della mia vita (\textit{forse}), con il quale ho condiviso due ottimi parziali di fisica e tanto tanto odio nei confronti della segreteria UniMi. Ringrazio \textbf{Yeger}, mio fratello alternativo, non contattatelo con applicazioni recenti perché è già tanto se ha gli SMS, conosciuto davanti alla ``fontana'' di Celoria 18 e da quel momento abbiamo condiviso ogni caffé preso la mattina prima delle lezioni, i pranzi da Luca, il treno, le feste, e la lista potrebbe andare avanti per molto molto tempo. Infine, ringrazio \textbf{Yon}, mio compagno dal primo giorno di tirocinio, abbiamo legato tantissimo nell'ultimo anno e sono molto contento di aver trovato una persona così solare, divertente e disponibile. \\

\noindent Non potevo chiedere compagni migliori, mi avete accolto dal primo momento e ora siete ancora qua, con me, per festeggiare questo traguardo. \\

\noindent Ringrazio \textbf{Gaetano} e \textbf{Charif}, gli unici amici che ho avuto a Matematica, grazie per tutte le ore passate in biblioteca a studiare per gli esami (\textit{passati 1 su 8}) e per le pizze in Piazza Leo tra una lezione e l'altra. Ringrazio anche tutte le persone che ho conosciuto grazie ai miei compagni di università, soprattutto \textbf{Samu}, con il quale ho condiviso una vacanza pazzissima in Croazia, \textbf{Anna}, \textbf{Elena} e \textbf{Grazia}, ma in realtà, anche no. \\

\noindent Ringrazio i miei compagni di classe delle superiori, quelli che sono rimasti in contatto con me anche dopo la maturità, con i quali mi sono incontrato ogni tanto negli anni per rivivere ``i bei vecchi tempi''. Ringrazio quindi \textbf{Botta}, \textbf{Ceres}, \textbf{Elena}, \textbf{Ferro}, \textbf{Gara}, \textbf{Mangio} (\textit{il terzo, unico e inimitabile}) e \textbf{Vaila}. Ringrazio anche \textbf{Antonio}, il Gargiu, che so che da lassù mi sta guardando e sta facendo il tifo per me. Sempre delle superiori voglio ringraziare il \textbf{Prof. Bellavita}, per avermi trasmesso la passione per l'Informatica, la \textbf{Prof.ssa Bolzoni} e, successivamente, la \textbf{Prof.ssa Delmari}, per avermi reso dipendente dalla Matematica, e il \textbf{Prof. Bardelli}, perché mi manda ogni anno gli auguri di Natale e Pasqua. \\
    
\noindent Ringrazio tutti i miei amici, compaesani e non, che mi conoscono da una vita e non sono ancora finiti in una clinica psichiatrica per colpa mia. Siete degli amici fantastici, ci siete sempre stati per me e spero di avervi per sempre nella mia vita. In questi anni mi siete stati vicini quando alcuni esami non andavano bene ed esultavate con me quando invece andavano bene, parte di questo traguardo è merito vostro. \\

\noindent Secondo round. Ringrazio \textbf{Andrea} e \textbf{Carol}, fedelissimi compagni di basket e sushi, quando esco con loro sono un po' il loro bambino visto che Andrea perde tutte le scommesse e mi paga sempre da mangiare. Ringrazio \textbf{Biga}, \textbf{Miriam}, \textbf{Adele} e \textbf{Rek}, un quartetto fantastico dove l'unico che si salva è il \textit{re del fuoco}, fedelissimi compagni di grest, viaggi in treno, grigliate e spero ancora tanto altro. Ringrazio \textbf{Arianna} e \textbf{Lisa}, il duo delle capre dove la prima ha la 104 e la seconda pure, grazie per non aver mai smesso di credere in me, anche quando ero molto giù e non vedevo una soluzione ai miei problemi. Ringrazio \textbf{Deep}, mio fratello, conosciuto assieme alle ultime citate, ci siamo presi una bella pausa di 4 anni ma appena ci siamo rivisti era come se nulla fosse successo; prima persona a credere in me come ``palestrato'', ha tifato per me all'inizio e alla fine del mio percorso universitario, ora fammi uno Yellowstone con doppie patatine. Ringrazio \textbf{Chiara} e \textbf{RSP}, coppia mancata perché a quanto pare Gesù attira di più della patata, con la prima ho legato di recente mentre con il secondo ho una relazione segreta da circa 4 anni. Ringrazio \textbf{Delia}, gymsis che ama fare lo step, non ci vediamo mai durante l'anno ma puntualmente al compleanno di Luca abbiamo il nostro discorso filosofico su come sta andando la nostra vita. Ringrazio \textbf{Davide} e \st{Elena} \textbf{Martina}, il mio migliore amico dalle elementari e la mia vicina da ormai 15 anni, un fidanzamento rocambolesco durante una festa molto movimentata, tante vacanze assieme, tante grigliate finite male, tante bellissime esperienze che spero non finiscano mai. Ringrazio \textbf{Francesca}, non ha la 104 ma almeno la 208, impedita con i computer e tutto ciò che esiste di tecnologico, ci conosciamo da secoli immemori e, se non fosse per la palestra, faremmo una fatica immane a vederci e riconoscerci per strada. Ringrazio \textbf{Laura} per aver tifato per me dal primo giorno che mi sono iscritto di nuovo in università, e anche se adesso siamo un po' litigati, sono sicuro che sta tifando ancora per me. Ringrazio \textbf{Luca Maestri}, conosciuto per puro caso una sera in un bar, da quel momento abbiamo condiviso feste, momenti in stazione a Lambrate, consigli sui corsi e tanto tanto altro. Ringrazio \textbf{Manpreet}, mio fratello che mi ha abbandonato dal nulla dopo la terza media ma che è uno dei pochi di quella classe che è rimasto con me, anche se ci sentiamo poche volte durante l'anno sono contentissimo della sua presenza, soprattutto a questa laurea. Ringrazio \textbf{Don Emanuele}, per tutte le volte che mi ha aiutato, supportato e ``impegnato'' le giornate tra muratori, contabilità e spostamento mobili. Ringrazio il \textbf{Gippe}, che appena ha saputo della mia iscrizione a Informatica ha subito cercato di indirizzarmi sui binari giusti e tifava per me ad ogni esame. Ringrazio tutti gli \textbf{animatori del Grest}, che durante la sessione estiva mi alleggerivano le giornate pesanti di studio facendomi divertire come solo loro sono capaci. Ringrazio tutti i miei gymbro della palestra di Treviglio, soprattutto \textbf{Francesco}, che non vede l'ora di avere una copia della mia tesi per fare una cosa che solo lui sa.  \\

\noindent Ringrazio i \textbf{Roggiani}, i miei fratelli, Luca (\textit{RR19}) e Mastro (\textit{RR17}), grazie per tutto quello che fate per me ogni giorno, per non abbandonarmi in palestra ad orari improponibili, per rendermi la persona più felice del mondo e per farmi sentire a casa ogni volta che sono con voi. Ringrazio il \textbf{Tunet}, che assieme al Ciapa mi liberava la mente ogni sera per 2h durante il periodo delle zone Power Ranger e mi ha tenuto compagnia per tutto il periodo universitario tra treni cancellati o perennemente in ritardo. Ringrazio \textbf{Vittorio}, unico (\textit{polipopi-ingegnere}) informatico che la mattina sul treno mi ascolta e non fa una faccia schifata perché non capisce quello che dico, autore del 99\% delle correzioni fatte nella mia tesi, \textbf{é} stato importantissimo nella parte finale del mio percorso. Ringrazio \textbf{Giacomo}, il mio gymbro, il mio cioccolatino ripieno, il mio pookie, grazie per non avermi abbandonato ogni volta che volevo andare ad allenarmi ad orari proibitivi, per aver ascoltato (\textit{e non aver capito}) ogni argomento che facevo a lezione o quello che facevo durante il tirocinio, per tutti gli sgarri fatti dopo 2h chiusi in palestra con Chiara che aspettava solo di tornare a casa, per tutti i concerti che fai ogni tre secondi dove crei ogni volta una parodia diversa, per essere il giocatore più scarso di Clash Royale, grazie veramente per tutto, \textit{you are my sunshine}. \\

\noindent Ringrazio \textbf{Martina}, la mia ragazza, grazie per esserci stata in questi ultimi anni, per il continuo supporto che mi dai ogni giorno, per sopportare ogni cosa stupida che faccio, per tutte le esperienze fantastiche che abbiamo fatto assieme, per aver ascoltato ogni mio audio di 25 minuti dove ripetevo allo sfinimento gli argomenti dei vari esami, per essermi stata vicina ogni volta che piangevo perché un esame non andava bene o accumulavo troppa ansia, per aver esultato ogni volta che ``HO SISTEMATO IL CODICE, ORA FUNZIONA'' (\textit{avevo dimenticato una virgola, non so programmare}), grazie per essere semplicemente te. \\

\noindent Infine, ringrazio la mia famiglia. Ringrazio \textbf{mamma Barbara} e \textbf{papà Marco} per tutte le volte che mi hanno accompagnato nelle stazioni della bassa cremasca per prendere il treno, per tutte le volte che mi hanno aiutato preparandomi il pranzo o la cena quando ero in ritardo con i miei impegni, per il continuo e immancabile supporto ogni volta che si avvicinava la sessione, per tutto quello che han fatto per me in questi 23 anni, per avermi dato una seconda opportunità dopo la brutta esperienza a Matematica, grazie veramente, so che molto spesso non lo dimostro, ma sono orgogliosissimo di avere due genitori così fantastici. Ringrazio mia sorella \textbf{Eva}, per tutte le volte che ha messo la musica a volume 100 quando dovevo studiare, per tutti i ``mi accompagni di qua?'' ogni volta che avevo un impegno immediato da sbrigare, per tutti i ``io esco alle 3 dalla disco, vieni a prendermi?'' ogni volta che volevo dormire, ma anche per tutti i bei momenti che abbiamo passato assieme negli ultimi anni. Vi ringrazio di cuore, vivere con me non è facile visto che sono molto chiuso e sto molto sulle mie, ma sappiate che vi voglio veramente bene e sono grato di ogni momento passato assieme. Ringrazio i nonni \textbf{Antonia}, \textbf{Antonio}, \textbf{Carolina} e la bisnonna \textbf{Carla}. Ringrazio gli zii \textbf{Annalisa}, \textbf{Tiziana}, \textbf{Wilma}, \textbf{Eleonora} e \textbf{Simone}, \textbf{Goffredo} e \textbf{Jessica}. E per concludere, ringrazio i fantastici cugini \textbf{Cristiano}, \textbf{Federico}, \textbf{Giorgia}, \textbf{Gianella}, \textbf{Elisa} e \textbf{Alessia}. \\

\noindent Ringrazio tutti di cuore, grazie per esserci stati, e grazie a tutti quelli che ci saranno nei miei prossimi traguardi.
