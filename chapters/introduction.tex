\chapter{Introduzione (COMPLETATO)}

\section{Obiettivi}

L'obiettivo del lavoro di tesi é testare la famiglia ASCON, vincitrice della gara per la standardizzazione della crittografia lightweight indetta dal NIST. L'attività di testing riguarda tutti gli algoritmi presenti nella suite ASCON ed é stata eseguita su tre dispositivi IoT fisici, presentati nel capitolo ``Testing e analisi''. Nello stesso capitolo é presente anche l'analisi delle performance di ogni algoritmo. Ogni dispositivo testato ha seguito un workflow preciso: \begin{enumerate}
    \item individuazione dell'architettura hardware a disposizione;
    \item scelta delle implementazione degli algoritmi da testare;
    \item compilazione di questi ultimi;
    \item raccolta e analisi dei tempi di esecuzione.
\end{enumerate}

\section{Struttura dell'elaborato}

L'elaborato inizia con una breve introduzione al mondo dell'IoT, definendo contesti di utilizzo e sfide, e alla crittografia lightweight, con l'analisi completa della famiglia ASCON. Successivamente viene presentato il capitolo riguardante l'attività di testing e analisi fatta sui dispositivi fisici. Infine, é presente un capitolo conclusivo che presenta un riassunto dei risultati ottenuti e possibili sviluppi futuri.
